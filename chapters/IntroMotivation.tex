\section{Motivation}
%\subsection{Subsection}
In numerical or computational science and more specifically numerical mathematics efficient discretization procedures are of crucial importance for different types of problems we encounter in various applications. All of these applications have multiple tasks in common, for example definition of sets of points known as grid generation, computation and evaluation of function values on these point, interpolation of function to estimate a value at an arbitrary point, or integrating or differentiating functions for solving differential equations using different schemes.\\
In particular we are interested in interpolation problem in our research certainly because it is the baseline problem to prove the combination technique can even be used for other more advanced problems. Through extensive search and review of previous related works, it seems that the base of our method comes from classical Richardson extrapolation\cite{Rude1994}. On the other hand, Russian mathematician Smolyak using the basic of Richardson extrapolation presented a method which he used for numerical integration. One can imagine this gave birth to a foundation of sparse grids \cite{smolyak63quadrature}.
The underlying idea of Richardson was to use of different discretizations with different resolution such that fine mesh approximations are fine-tuned recursively by approximations on coarser levels. This idea is really close to multigrid methods and uses the same hierarchical structure.  Such strategies, where multiple distinct grids participate to define a combined  result, are generally called extrapolation methods such as Classical Richardson extrapolation.\\
Classical Richardson extrapolation can be extended and generalized in many ways. If this generalization is being done using  mesh widths in different dimensions, coordinate directions as the major playing parameters it will lead to the so-called multivariate extrapolation\cite{Rude1994}. Note that combination extrapolation can be interpreted as a special case of multivariate extrapolation \cite{Griebel1992, Rude92extrapolationand}. Even further, it has been shown that one case of combination extrapolation is exactly the combination technique proposed in \cite{Griebel1992b}. The primary idea of combinaton technique is identical for multilevel splitting of finite element spaces and it is to replace hierarchical bases of the finite element spaces instead of the usual nodal ones \cite{Yserentant1986}. As explained earlier various names, such as (discrete) \textcolor{red}{blending method} \cite{}[21], \textcolor{red}{Boolean method} \cite{}[11], sparse grid method \cite{Gerstner1998}, technique of hyperbolic crosses \cite{Griebel1998}, or splitting extrapolation \cite{Zumbusch2000} are practically interchangeable.  %come from ref of 11
As the name is just a representation for the idea, we stick to the combination technique introduced by Zenger et. al in  \cite{Griebel1992b}\\
The advantages of this method discussed extensively in literature are as follows:

\begin{enumerate}
	\item \textbf{Reducing the number of grid points:} The combination technique borrows this characteristic from the properties of sparse grids. In comparison to the full grid approach, the number of grid points (unknowns) can be drastically reduced.
	\item \textbf{Inherently parallelizable:} Since the grids to be combined are usually independent of each other, it makes it easy to imagine how it can be parallelized using MIMD\cite{Griebel1992a}, Network\cite{Griebel1992} or GPGPUs\cite{Gaikwad2009}. The coarse grain parallelism of the combination method makes it perfectly suited for MIMD parallel computers and distributed system on workstation networks. Parallelization of combination technique supports both modularity and portability by separation of sequential modules. The gain is expected to be even more dramatic for higher dimensions. However, the collection of the results is not trivial as we need a strategy like trees to combine the solutions.
	\item \textbf{Simplicity of the concept:} its framework allows the usage and integration of existing solvers and methods\cite{Bungartz1994}.
	\item \textbf{Good accuracy:} the combination technique usually doesn't require as much storage space and computing time as the usual full grid but achieves nearly the same accuracy\cite{Griebel1995}.
    \item \textbf{Robustness:} Later it will be shown how this method can be exploited to various problems under certain conditions. Specially compared to sparse grids since it doesn't involves hierarchical, tree-like data structures and special algorithms for both the discretization and the solution it can be used in conjunction with conventional solvers.
    \item \textbf{Speed of convergence:} its speed of convergence does not depend on the regularity properties of the considered boundary value problem or the refinement resolution\cite{Yserentant1986}.
    \item \textbf{Optimal computational complexity:} The computational complexity of combination technique is almost the same as the conventional multigrid methods but without their restrictions\cite{Yserentant1986}.
\end{enumerate}

Given these many advantages we are going to proceed and further improve the combination technique towards spatial adaptivity. Note that only major draw back which can not be ignored here is the error and convergence analysis. Later we see for various applications, there are certain conditions needed to be ensured of convergence.
