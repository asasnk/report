\chapter{Related Works}
\label{chapter:relatedworks}
Comprehensive literature study will be presented here. The author tries to categorize the researches in separate sections. By doing so, faster access to the different applications for a seeking scientist can be achieved. While each work is referenced, we try to give major important remarks and conclusions of that research so that it can be a good preparation and start for deeper search.
\section{Works on Complicated Domains} 
The combination technique is not restricted to the unit square. Successfully treatment of problems on distorted quadrilaterals, triangles, polygonal boundaries domain have been investigated by Griebel. The solution of problems with a nonlinear operator and partial differential equations like the Stokes and Navier-Stokes equations have been presented. \cite{Griebel1992a, Griebel:1993*3}\\
It has been shown that to some extent the combination technique works even in the case of non-smooth solutions like complicated domains explained earlier but replacement of ${h}^2_i$ and ${h}^2_j$ by ${h}^\alpha_i$ , ${h}^\beta_j$ with appropriate $\alpha$ and $\beta$ in the given problem is required. Because of the properties of the combination technique the major error terms still cancel each other this way. However, for the problems with severe singularities, the appropriate combination of adaptively refined grids is recommended\cite{Griebel1992a}. This results that our research can also be applied to this type of problems.
\section{Works on Error Analysis and Convergence} 
Although, the implementation of the combination technique seems trivial, General convergence of the method cannot be proved with usual standard arguments from finite element theory. Therefore, there has been a lot of works which we present in chronological order to give insight on what has been done. Recent studies \cite{Reisinger2013} uses more general convergence scheme with definition of error bounds but still there are plenty of assumptions there.
Convergencea  and error anlysys of the combination technique for the finite element solution of Poisson's equation and 2nd-order elliptic differential equations which is general form of Poisson equation has been investigated in early works and it has been the base for error analysis of the combination technique.\cite{Pflaum1993, Pflaum1997}\\
Bungartz et al have investigated a model problem of Laplace equation on the unit square with a Dirichlet boundary function  based on finite difference and Fourier techniques on a pointwise manner \cite{Bungartz1994}. They proposed it there exist an error splitting if the Dirichlet boundary function satisfies a certain smoothness requirement i.e. Fourier coefficients in case of Laplace equation. Multiple numerical solutions for both smooth and nonsmooth boundary functions have been studied the convergence of the pointwise error.\cite{Bungartz1994}\\
A technique to analyze the convergence rate of the combination technique applied to general second order elliptic differential equations in two dimensions and its proof for Poisson's equation convergences in arbitrary dimensions is later inspected by Pflaum \cite{Pflaum1999}. The difference to his early work is the removal of requirement that the normal derivative of some coefficients should be zero at the boundary. The significant consequence is the proof the convergence of the combination solution on a complicated domain or more precisely curvilinear bounded domain. The scheme is to divide the curvilinear bounded domain in several blocks and to transform each block onto the unit square. \cite{Pflaum1999}\\
Further improvement has been done which is the base for the Introduction of optimized combination technique. Since we hardly know about the reasons of this effectiveness or divergence criteria of the combination technique. A technique which inherently uses the error terms has been introduced which is the so-called optimized combination technique. This is based on the fact that the combination technique gives an exact result in the case of a projection into a sparse grid space if their partial projections commute. They have analysed the performance of the combination technique in a projection framework and used the C/S decomposition. Based on that analysis modified optimal combination coefficients are derived and substantially expand the applicability and performance of the combination technique.\cite{Hegland2007}\\
Most recently in the error analysis of combination technique has been shown that it requires derivation of a specific multivariate error expansions on Cartesian grids and for linear difference schemes through an error correction technique. By this an error formulae will be derived to use for analysis of the convergence. Note that its dependence on dimension and smoothness in case of linear elliptic and parabolic problems has been on the focus. Finally, introduction of a new framework to analyze error bounds for general difference schemes in arbitrary dimensions is given. \cite{Reisinger2013}\\ 
\section{Works on Parallel Environment}
Coarse grain parallelism which basically is kind of decomposition of tasks for the combination method makes it prefectly suited for MIMD parallel computers and distributed processing on workstation networks. There has been early studies parallel for the solution of elliptic partial differential equations on MIMD structured computers and parallel sparse grid preconditioning or solution of partial differential equations on Workstation networks \cite{Griebel1992, Griebel1992a}.\\
Concept of parallelization can be applied to different parts of solutions, for instance, the parallelization of the basic iterative method, the parallelization of the preconditioning step and so on. However, most types of preconditioners are not parallelizable that efficiently. They require modifications of the numerical algorithms resulting in slower convergence rates. and well parallelizable preconditioner using combination technique has been introduced by Griebel. They have compared their method with preconditioners inefficiently parallelizable and with preconditioner like classical multigrid which are well parallelizable but do not possess the natural parallel characteristic of the combination method. In contrast to these techniques, the parallelism in the combination technique is more explicit. Similar to the domain decomposition approach, the combination method possesses a simple parallelization potential, because all the subproblems are independent\cite{Griebel1992}.\\
Further parallel experiments on MIMD-machines and networks, however, have shown that it is insufficient to achieve substantially better efficiency rates for combination method. Therefore idea of complex or advanced load balancing strategy is necessary to exploit the benefits of massive parallel systems equipped with quick communication hardware.\cite{Griebel1992} \\
Till that point of time, no comparison has been done with usual sparse grid methods, simply becasue the parallelization of an sparse grid code usually is non-trivial and requires a substantial effort on coding. Early works on this regard to compare it with combination technique is investigated in\cite{Zumbusch2000}.
A parallel version of a finite difference discretization of paratial differential on arbitrary, adaptively refined sparse grids is proposed. The efficient parallelisation is based on a dynamic load-balancing approach with space-filling curves. with applications can be in higher-dimensional problems such as financial engineering, in quantum physics, in statistical physics and in general relativity. Presented issue there is usually hierarchies of refined grids in neighbour nodes may reside on different processors so it needs to be managed. One solution can be creation and updating of appropriate ghost nodes on a communication operation. The space-filling curve is simply a unique mapping of nodes to processors so it immediately shows which processor needs to be communicated with \cite{Zumbusch2000}. Relevently, A load model for linear initial value runs with GENE is introduced for effective load balancing for the combination technique in \cite{Heene2014}. \\
Another interesting concept is the idea of using GPGPUs also known as multicore progreamming. In multi dimensional option pricing problems of computational finance the sparse grid combination technique can be a practical tool to solve arising PDEs. Using Hierarchization leads to linear systems smaller in size compared to standard finite element or finite difference discretization methods. Excessive demands memory for direct methods which challenges the iterative methods suggests the usage of massive parallelism of general purpose Graphics Processing Units (GPGPU)s. It also requires proper data structures and efficient implementation of iterative solvers. Performance analysis and the scalability of combination technique based solvers on the NVIDIAs CUDA platform compared to CPUs for certain applications shows promising results. becasue of locality and linearity properties.\cite{Gaikwad2009}\\
Advanced idea of hierarchization as preprocessing step to facilitate the communication needed for the combination technique has been presented in \cite{Hupp2013}. The derived Parallel hierarchization algorithm outperforms the baseline drastically and achieves good performance. The algorithm needs iterative hierarchization and dehierarchization. (For further details please check \cite{Hupp2013}) \\
Lastly, in discussion of fault tolerance methods we know extreme scale computing usually leads to the increase in probability of soft and hard faults. Parallel fault tolerant algorithms with modification of sparse grid combination method is in focus to solve partial differential equations in the presence of faults. It modifies combination formula to accommodate the loss of few component grids. A prototype implementation within a MapReduce framework using the dynamic process features and asynchronous message passing of MPI is presented.\cite{Larson2013}. 
\section{Works on Fluid Mechanics, Heat Transfer and other PDEs} 
Arguably, one of the most important application areas for combination technique can be in solution for partial differential equations with high dimension, high computation complexity or high memory demands regrading high order number of grid points. As first demonstration of the advantages of combination approach investigation on modal problem has been done in two dimensional cases \cite{Griebel1992b}. Modal problems are as follows:
\begin{enumerate}
\item Smooth Solution
\item Singular Solution
\item Boundary layer (solution of combination technique is equal to full grid, in general holds for solution only depends on one direction or boundary layer)
\item Distorted quadrilateral and triangular elements
\item Nonlinear heat transfer
\end{enumerate}
In case of nonlinear heat transfer, the combination technique produces good solutions. The error quotient is major factor to check here. it has been shown that the efficiency, the degrees of freedom, the run time and the achieved accuracy for the combination technique is far better than the full grid approach. Also note that additional Newton iterations seems to be a remedy the approcah even further \cite{Griebel1992b}.\\
First studies of the sparse grid combination technique as an efficient method for the solution of fluid dynamics problems has been done in \cite{Griebel1995}. it shows the numerical experiments for the application of the combination method to CFD problems e.g. two-dimensional laminar flow problems with moderate Reynolds numbers. The research is based on the fact that implementation of the combination technique can be based on any black-box solver. Usually, fluid dynamics problems have to be solved on rather complex domains, thus, a reasonable approach is to decompose the domain into blocks to handle the problem. Obviously, the combination technique works on such block structured domains as well as complicated domains like a graded grids. Since the sparse grid methods are highly economical on storage requirements, given that they produce a fairly accurate solution, the usage is gihly recommended \cite{Griebel1995}.\\
In \cite{Griebel1999}, promising numerical results are presented for the combination technique applied to a constant coefficient advection equation for four test cases of. Their work differs from \cite{Griebel1999} in that it also presents error estimates \cite{Lastdrager2000}.
\begin{enumerate}
\item Horizontal advection
\item Diagonal advection 
\item Time dependent advection
\item The Molenkamp-Crowley test case
\end{enumerate}
Similar to heat transfer problem explained above, in \cite{Lastdrager2001} a significant progress in the numerical simulation of systems of partial differential equations in the advection diffusion reaction equations of large-scale transport problems in the modeling of pollution of the atmosphere, surface water and ground water has been achieved. Since these models are three dimensional and modeling transport and chemical exchange over long time periods requires very efficient algorithms computational capacity is a restricting factor. For example in simulation of global air pollution huge numbers of grid points is needed, each of which many calculations must be carried out in. The application of sparse grid combination techniques might be a solution. However, in their research they only considered pure advection and left the diffusion and reaction processes to future research\cite{Lastdrager2001}.\\
More recently, a convection diffusion problems on the conventional unit square has been investigated and observed using a sparse grid Galerkin finite element method in\cite{Franz2009}.\\
Lastly, application to probablistic equations like the stochastic collocation method based is on the horizon. For instance, an anisotropic sparse grid solution is an important tool to solve partial differential equations with random input data\cite{Erhel2015}.
\section{Works on Numerical Integration} 
Based on original idea of Smolyak there has been some similar work in numerical integrations. For example \cite{Gerstner1998} they review and compare existing algorithms for the numerical integration with the the ones of multivariate functions over multi-dimensional cubes using several variants of the sparse grid method.\cite{Gerstner1998, smolyak63quadrature}\\
Multivariate integrals arise in many application fields, such as statistical mechanics, computational finance and discretization of partial differential and integral equations or the numerical computation of path integrals. The so-called curse of dimension are also in play in these conventional numerical algorithms for computation of integrals there. So the rectifying remedy of sparse grid combination method is used in \cite{Gerstner1998}.
\section{Works on Regression Problem}
In the context of regression or basically fitting the function to given values there has been some works, in \cite{Hegland2002}, they compared the iterative algorithm for multidimensional sparse grid regression with penalty and showed the improving performance compared to iterative methods based on the combination technique.\\
The combiation technique approach shows instabilities in some situations and is not guaranteed to converge specially with higher discretization levels. As stated before the optimized combination technique can repairs these instabilities. based on the fact that combination coefficients also depend on the function to be reconstructed, thus a nonlinear solution. It has been shown that the computational complexity of the optimized method still scales in linear manner to the number of data.\cite{Garcke2006}\\
In \cite{Garcke2009} there has been investigation in theory and expreiment of the reason why combination technique solution for regularized least squares fitting is not as effective as it is in the case of elliptic partial differential equations. Their arguement is that this is due to the irregular and random data distribution, and dependency of number of data to the grid resolution. They note that overfitting can arises when the mesh size goes to zero. So they conclude an optimal combination coefficients can prevent the amplification of the sampling noise present.\\
\section{Works on Data Mining and Machine Learning}
Data is currently produced with a huge rate. The issues in data processing sciences are mainly the ncreasing amount of data recorded and also the increasing complexity of these data. Application of analysis of image, multimedia, or spatial data are some examples of it.. An important task in data modelling is to develope prediction models or functional relationships between their selected features. The so-called curse of dimensionality arises when one want to identify such predictive models. In \cite{Hegland2003}, Hegland discusses, how to choose the function spaces with an iterative approach to increase complexity of functions. His approach is to use adaptive complexity management closely related to the Analysis of Variance, also known as ANOVA decomposition. As the sparse grid approximations is good framework in this regards it has been combined with regression trees and multivariate splines to analyze the complexity of solution\\
Similar to the idea above the generative dimensionality reduction methods in machine learning has been investigated in \cite{Bohn2016}. They propse a framework to build a  mapping from a lower dimensional space problem to higher dimensional data space and vice versa to achieve a dimension adaptive sparse grid reduction method. The reason to do so in data analysis is becasue some directions are play more important role than the others than others and it is reasonable to refine the underlying discretization space only in these directions based on the value of reconstruction error in that dimension.\\
\section{Works on Eigenvalue Problems}
In computational physics the concept of eigenvalue problem arise for example for the Born-Oppenheimer approximation of the stationary Schr\"odinger equation for atoms. A discrete eigenvalue problem is based on finite element discretization of the problem. The authors of \cite{Garcke2007} propose to use optimized combination technique to be used for the solution of this problem but directly applying to eigenvalue problems is not possible. Their remedy is to use partial solutions as ansatz functions and reconstruct the projection of the optimized combination technique as a Galerkin-approach\cite{Garcke2007}.\\
Similarly, in the context of hot fusion plasmas there is a five-dimensional gyro-kinetic equations. As five dimensional problem in plasma needs a lot grid points and it is exponentially grown becasue of factor five one will encounter the curse of dimensionality. However as shown before the combination technique can be applied to this eigen-value problem for the gyro-kinetic code GENE\cite{Kowitz2013}.
\section{Works on Adaptive Methods}
This section is in major importance since it is closely related to the work of author of this project, in that they are also adaptive methods. First major study in this regard has been done by Griebel \cite{Griebel1998}.
Adaptive Sparse Grid Multilevel Methods for Elliptic PDEs Based on Finite Differences: They have investigated how to use standard operations between two functions, i.e. addition or subtraction, scalar multiplication, and Division. This is done by transformation of values to nodal basis and accumulation in nodal basis and then returning to hierarchical basis representation. By this setup we can estimate the error indicators and perform adaptive multilevel treatment of PDEs. First hand usage of hash table benefits in the case of adaptive multilevel treatment are also another important part of the research\cite{Griebel1998}.\\
Another important research has been in the context of machine learning and regression \cite{Garcke2007a}. Note that we included this work here rather than in regression section simply because of importance of their work in correspondence with the research at hand. It has been proposed how to use hierarchy of generalized sparse grids and choose the partial functions with adaptive iterative procedure. By doing so, one can pick out features insignificant to the prediction and thus the adaptivity.



