\section{Goals}
As you will observe, comprehensive literature review has been made on this subject including areas such as error analysis and convergence, solution in complicated domains, numerical integration, regression and data fitting problem, eigenvalue problems, parallelization schemes, data mining and machine learning, fluid mechanics, heat transfer and some stochastic or random partial differential equations, and finally adaptive sparse grid methods been presented throughout years. There hasn't been any attention to possibility of introducing a spatial adaptive combination technique. Perhaps, the reason is that it requires more complex data structure, projection and interpolation methods to achieve it. However in this research we will tackle this idea with a greedy but simple algorithm. Details of the algorithm is explained later in the implementation chapter.